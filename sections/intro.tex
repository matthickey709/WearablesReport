Wearables as defined by Technopedia are technologies that are worn on
the body that contain various sensors that can record health and fitness
information, or take movement input data in real-time \cite{technopedia_defn}.
The market for this technology has expanded rapidly in recent years, with the
wearable market being worth \$19 billion in 2015, and expected to expand to \$57 billion
by 2022 \cite{market_growth}. This growth rate can be attributed to the fact that it is a novel
technology just getting past the early adoption phase, but this technology is also improving
at an impressive pace each year. The 2010s have seen advances in lower-powered processors with 
a smaller footprint that allow wearable devices to become much more powerful. With improvements in
small, powerful processors, it allows wearables to have more functionality, and focus less on
designing the wearable around the electronics inside \cite{wearable_rev}. Clearly, this demonstrates
the design requirement for low-power and small components to architects of wearable technology.

While there are many types of wearables on the market in present-day 2019, this report will
focus on two types of wearable technology: smartwatches, which record movement data
for health and fitness purposes; and virtual reality (VR) headsets, which process
movement data in real time for immersive digital experiences.