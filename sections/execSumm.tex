Wearable technology, which is technology worn on the body that processes information,
is a rapidly expanding industry. With thanks to the meteoric improvement in
low-footprint, energy-efficient processors, wearables are becoming more powerful,
and more attractive to consumers. Two major classes of wearable technology are analyzed
throughout this report: smartwatches; and virtual reality headsets.

Smartwatches are wearables worn on the wrist, like a normal watch, and are normally
used to record health and fitness data. Two different smartwatches are analyzed
for their architectures: the Apple Watch Series 5; and the Garmin Forerunner 235.
These two watches employ different architectures, with the Apple Watch being more
powerful from a compute power standpoint, and the Forerunner being a much more energy
efficient smartwatch.

Virtual reality headsets are wearables that allow the user to be immersed within a
virtual environment with a device worn on the head that covers the eyes with displays.
The headset takes in user's movement data, and processes this data to update the virtual
environment with movements that make the user feel like they are part of that virtual
world. The Oculus Quest all-in-one headset was analyzed for its architecture. After
inspection, it is shown that the Quest uses the same system on a chip, and storage
and memory specifications as the Google Pixel 2 smartphone. This shows that the
architecture decisions made for the headset were to mimic those of a smartphone, with
the main difference being more intense processing and a shorter battery life.

After analyzing the case studies of the smartwatch and the virtual reality headset,
it is clear that the main architecture decision of wearables is the trade-off between
performance and power consumption. The exact place where this trade-off is made depends
entirely on the final use case of the wearable being designed.