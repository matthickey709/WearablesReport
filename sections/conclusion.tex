This report has offered an overview of two distinct types of wearables,
smartwatches and virtual reality headsets. These two wearable families
cover two different purposes for wearable technology: data collection
from your body's movement and health statistics; and using data collected
by monitoring movement, processing it in real time to provide the user
with an immersive virtual experience.

Analysis of the Apple Watch and Garmin Forerunner, which were released
in a similar time period, show that choice for computer architecture
design is impacted by the desired use case for that product. The Apple
Watch has impressive hardware, and offers a rich user experience with
a faster clock speed, and a 64-bit architecture versus the Forerunner's
slower clock speed and 32-bit architecture. In fact, the maximum clock
speed of the Apple Watch's processor is over 33 times faster than the
maximum clock speed of the Forerunner, illustrating the performance difference 
between these two devices. Despite the Apple Watch's superior performance, the 
Forerunner's battery lasts about 9 times longer than the Apple Watch battery
(ignoring GPS usage on both devices). This shows how Apple chose a performance-focused
architecture, while Garmin chose an efficiency-focused architecture, which target
users with different needs.

The Oculus Quest is a wearable, just like both smartwatches analyzed in this report,
however instead of being used as a device worn at all times to monitor health and
fitness statistics, it is a device worn when the user wants to experience virtual reality.
This wearable tracks user's movements and uses them as inputs to update the virtual
environment that the user is currently experiencing. The Oculus Quest, being an
all-in-one, or standalone headset needs to have a well-designed architecture
to offer powerful graphics rendering after taking inputs from the user's movements,
all while being as efficient as possible due to the device running on battery power.
Therefore, the architecture reflects a device that must possess powerful processing
ability, but must be efficient on battery power - a smartphone. The SoC and storage/memory
specification used in the Oculus Quest are identical to those used in the Google Pixel
2 smartphone. The main difference is the Oculus Quest is executing more compute-heavy tasks,
so its battery life is shorter than that of the Pixel 2.

As was shown through case studies of various wearable devices, wearable architectures
must make trade-offs between performance and power, and how much an architecture
leans one way or the other depends on the use case for the device employing that
architecture.